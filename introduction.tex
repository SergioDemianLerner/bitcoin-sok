\section{Introduction}\label{intro}

Bitcoin is a digital currency that has recently become very popular. \anote{more backstory?} It has received some attention from researchers, who have analyzed aspects of its operation and proposed extensions and solutions to attacks. A large and vibrant community of open-source developers and other participants have stewarded the system, proposing (and implementing) numerous extensions and modifications as well. Some of these have been implemented as alternate protocols, and compete with Bitcoin in a larger ecosystem of so-called ``cryptocurrencies'', while others have been adopted into Bitcoin itself (which remains, as of this date, far and away the most popular). The primary goal of this paper is to provide a framework for understanding and evaluating points in the space of possible designs for Bitcoin and related systems.

Bitcoin's goals are nominally focused on payments and the maintenance of a virtual currency - a user who owns a quantity of the currency may transfer it to someone else, typically in exchange for goods or services. This goal itself is not especially novel or difficult, especially if it's acceptable to rely on trustworthy servers and administrators. Digital currencies like Linden Dollars and World-of-Warcraft gold already provide this functionality; so do payment systems such as Paypal or debit cards (denominated in ordinary state-administrated currency, like the US Dollar or the Euro). What's remarkable about Bitcoin is that it aims to do this in a difficult setting {\em without any trusted parties} and {\em without pre-assumed identities among the participants}. Instead, the premise of Bitcoin is that greed is predictable\anote{incentives can align?} - given the right incentive mechanism, the network can encourage enough participation from the anonymous public to withstand any malicious but resource-constrained attacker. \anote{Basically it's the difficult model that makes Bitcoin interesting, not the money itself}

The contributions of Bitcoin are quite surprising here. The general problem of consensus in a distributed system is impossible in an anonymous network. Cryptographic approaches to secure multi-party computing rely on pre-established identities or a trusted PKI. Incentive schemes for p2p networks have typically assumed an external bank exists, rather than the ambitious approach of implementing money from scratch internally. Even approaches to cryptographic cash (ecash, micromint) rely at least on a trusted bank or issuer. Bitcoin has survived its first five years without suffering any critical attack - despite the clear temptation of breaking it in a way that steals the money - although it has not yet been given a formal theoretical foundation. The original whitepaper introducing Bitcoin sketched a proof that it is secure in an ``honest majority'' distributed system model, although this doesn't illuminate under what conditions, if any, the built-in incentive scheme {\em leads} to a sufficient amount of correct participation. In this paper, we do not try to provide a theoretical model of Bitcoin. However, in providing a more abstract understanding of Bitcoin, we believe our contribution can help inspire a theoretical development. \anote{The gap in theoretical basis makes it interesting, and is part of our motivation.}

If Bitcoin is successful in agreeing on a transaction history and replicated state machine, then it solves a much more general problem than merely transmitting payments. Indeed, there have been many proposals have been for more complex financial transactions than simply transmitting money, and there have even been proposals for non-financial transactions. One key to our understanding of the design space of cryptocurrency is the notion of ``Bitcoin as a platform'' for transactions involving arbitrary rules. The abstract description we arrive at, surprisingly, does not include a particular definition of money as a goal - instead, money shows up as part of a particular solution.

\subsection{Related Work}

- Other digital currencies, centralized like QCoin, Linden dollars. Other payment systems, like Paypal, debit card systems, etc.

- Quick summary of major academic work on Bitcoin, including extensions of Bitcoin, such as zerocoin. Other surveys or introductions to bitcoin, such as the bitcoin primary, etc. Analysis of Bitcoin, such as Kroll, et al.

- Other systems for distributed consensus, Byzantine impostors, homonyms, FLP. Preliminary approaches to ecash, chaum, micromint, hashcash.
Proof-of-work, client puzzles.

- Cypherpunks, Wei Dai, Nick Szabo, etc.
