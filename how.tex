\section{An Introduction to Bitcoin}
\anote{There are a few conflicting goals in this section. One is to introduce the basic terminology and concrete details about how the present Bitcoin works. Another is to introduce the key abstractions we'll use to understand Bitcoin variants. Another is to understand Bitcoin in context.}


\subsection{An Alternate History of Bitcoin in Steps}
\anote{The purpose of this section is to explain what's new about Bitcoin, and how it might have been derived systematically as an improvement to other approaches.}

The goal is to have basic checking account functionality for Alice and Bob.

\paragraph{Step 1: Imagine there’s a trusted third party (TTP)}
 Each participant is (somehow) assigned a single key pair which also acts as an address. The TTP allows anyone to send it messages and gives all participants a consistent view. Each participant (i.e., address) has some balance. We will defer the question of initial balances and where these balances come from.

If Alice (address $A$) wants to send Bob (address $B$) value $v$, she sends to the TTP a signed statement, called a transaction, saying “address $A$ transfers value $v$ to address $B$.” The TTP verifies each transaction (checks that $v ≤ balanceA$) and ignores the ones that don’t verify.

\paragraph{Step 2: Removing the TTP}

If Alice (address $A$) wants to send Bob (address $B$) value $v$, she broadcasts a signed statement to all other participants saying “address $A$ transfers value $v$ to address $B$.” Once every 10 minutes, the participants themselves (replacing the TTP) conduct a protocol to a) propose a batch of valid transactions, called a block, and b) vote on these proposals to select one.

Well known voting protocols have been studied, under some standard assumptions. First, it is assumed that a majority of the participants are honest. Second, it’s assumed that honest parties are able to broadcast synchronously to every other participant. Dishonest participants, on the other hand, are assumed to be able to equivocate, sending one message to some participants and a different message (or nothing) to others. Underlying all this is the fundamental assumption that there is a fixed set of participants that is known to everyone.

\footnote{For efficiency, there are randomized protocols.\anote{survey of randomized consensus, breaking $n^2$ barrier}}

\paragraph{Step 3: Removing limits on addresses}

In reality, participants don’t have preassigned identities, so the assumption that there is a fixed set of participants is untenable. Therefore we replace that assumption with a much weaker one: Imagine that there’s a magic token that lands on a random participant (chosen uniformly) at regular intervals. \footnote{The magic token is recognizable and unforgeable: if a participant receives a message from a participant holding the token, she knows that the sender did in fact hold the token.}

As before, participants broadcast each transaction to all other participants. When a participant gets the token she collects all the transactions she’s seen (in the most recent token interval), verifies them, and broadcasts the block.

\anote{Description of the hash chain data structure, how participants process blocks, longest chain rule, and how this leads to probabilistic consensus if the majority of the identities are honest.}

Note that participants can now create key pairs for themselves -- as many as they like. We have severed the link between participants and addresses. The only assumption is that the token selects participants uniformly at random, and hence selects an honest participant with a probability greater than $\frac{1}{2}$.

\paragraph{Step 4: Removing the token: proof-of-work}

Participants compete for possession of the token by racing to solve a proof-of-work based on hashing. Thus, participants acquire the token with a probability proportional to the fraction of Bitcoin’s hash power that they control. \footnote{there is a random nonce, so that each participant works on a different part of the problem space.} Attaching the proof-of-work to the block proves possession of the token, which is purely imaginary.

Now, instead of the majority of participants being honest, we require that the majority of hash power be controlled by honest participants. Except for that change, the argument for consensus in Step 3 holds here as well.

\paragraph{Step 5: Incentives}

To better justify the assumption that a majority of participants (by CPU power) are honest, we provide an incentive for honest behavior. This is done by adding value to the balance of an address chosen by each participant that finds a proof-of-work solution. \footnote{This is also the mechanism by which the values money are initially distributed, in an arguably fair way.} \footnote{Everything described so far is merely an abstract system for transferring and tracking balances. For the incentive system to work, these balances must be interpreted as money.}

This suggests a bootstrapping argument. If the transaction log is secure (has integrity and availability), then it will be useful as money, and therefore value added to an account makes a worthwhile incentive. This in turn will encourage a large amount of participation, which ensures the transaction log is secure.

This appears to be a virtuous cycle. On the other hand, neither system can exist without the other. If the integrity or the availability of the transaction log can be easily attacked, then we cannot expect it to maintain much value as a currency. Likewise, if the currency is not valuable as money, then it will not be an effective incentive for encouraging sufficient participation to prevent successful attacks.


\subsection{Law, Log, and Index: An Abstraction of Bitcoin as a Platform}

Abstractly speaking, what functionality does Bitcoin -- or more importantly, the innovations embodied by it -- provide? Originally presented as a mechanism for online ``payments,'' in the abstract it is a much more versatile platform.

\begin{enumerate}
\item (Log) A sequential broadcast medium. Every user eventually (in a short amount of time, with high probability) agrees on a global sequential log of messages (and every user receives every message in its entirety). Any user can publish a message, and it will be included in the log within a short time (although, a monetary fee may be required for timely service).
\item (Index) Participants on the network maintain a globally replicated storage index. Every message in the broadcast log is processed, and in some cases causes updates to the index. These indexes can be relied on as concise summaries of the public broadcast record.
\item (Law) The broadcast log may additionally contain data that reflecting the result of computations, including queries to the index.
\end{enumerate}

This framework can be used to understand the design differences between many Bitcoin-related proposals, all of which are special cases of the general structure. As an example, the standard usage of Bitcoin can be described as the following instance:
\begin{enumerate}
\item The broadcast log is used to publish signed messages called ``transactions'', which reflect an intent to transfer quantities of currency from one account to another. Quantities of the currency may also be set aside as fees.
\item Participants maintain a \emph{ledger} containing a mapping between every quantity of currency and the public key of its current owner. This index must be updated to reflect the new balance after every transaction.
\item Transactions are considered ``valid'' only if the attached signatures are valid with respect to the public keys in the current ledger, and if they spend a reasonable amount of money. In fact, only transactions that are valid according to these rules may be included in the log (although arbitrary data --- including invalid transactions --- may still be published as auxiliary data contained within a valid transaction).
\end{enumerate}
All three of these components are necessary, and complement each other. The sequential log is used to ensure that no two transactions double-spend the same quantity of currency. On the other hand, since only valid transactions carry fees, the index is used to ensure that data needed to validate transactions is readily available.

Since only valid transactions are included in the log, end-users (such as merchants with mobile devices) need only check that a transaction is included in order to confirm a payment. Although the log necessarily grows without bound, the mutable index does not; therefore nodes can ``prune'' old data from their main disks. These optimizations are discussed in more detail later on.

Most proposals for Bitcoin-like variations, including overlay currencies such as Mastercoin and Colored Coins as well as more ambitious altcoins like Ethereum, can be described in a similar way as we will show in Section~anote{refer}. Additionally, these key components provide a basis for evaluating the resource complexity of various schemes and use-cases.

The view of Bitcoin as a platform is that the currency is only inherently useful to the extent that it allows you to pay for transaction fees. In this sense it is comparable to postage stamps.


\subsection{System Attack Model and Consensus Requirements}

The Bitcoin mining consensus proof from the whitepaper, assuming a majority of the hash power is ``honest''. Bitcoin's solution to the consensus problem is novel, although the proof known so far is for the wrong model (honest rather than altruistic). Given a set of rules, in a distributed system, the problem remains of how to choose a correct order of operations.

Although, so far, the economic incentive system seems to work, there are only heuristics and no clear model. As best we can tell, the reasoning is circular. Incentive system is needed to make the system secure. The built-in currency that rewards participants has to be valuable for the incentive to be meaningful. The currency is valuable only if the system is secure.\footnote{For example, when there have been even temporary lapses in service or bugs announced, the price of Bitcoin has fallen~\cite{bitcoin-fork-news}.} Attacks are discussed further in Section\anote{refer}.

Bitcoin's assumptions are weaker than in a standard distributed system - most notably, there are no pre-established identities. On the other hand, the assumptions are about the rational preferences of population - specifically that they respond to incentives, even when the incentives are inconclusive. Seems to be circular?

Stabilizing consistency is weaker than a typical distributed system. In a typical distributed system, you receive an acknowledgment at some finite time that the transaction has been committed. In Bitcoin, there is no such acknowledgment, and users must user their own discretion about how long to wait to consider a transaction committed. \anote{This is partially a historical note, meant to clarify how Bitcoin's version of consensus/broadcast relates to more standard notions}

A typical definition of fairness or liveness would specify that *any* transaction should eventually accepted, without reference to adequate payment. However, since Bitcoin operates in a model with no established identities, this would be impossible due to the ability for an attacker to create sybil identities and flood the system.

\subsection{Concrete Details}
- Bitcoin's underlying mechanism.

- Mining, transaction propagation.

- Introduction of currency. Use of exchanges, market places. Financial ecosystem.

\subsection{How Changes are Applied}

- Discussion of governance and the need for social out-of-band consensus to agree on rule changes. Soft forks, hard forks, and policy. Very few hard forks in history.

- Discussion on the forums and mailing list that provide.

- Ability for altcoins to develop.

\subsection{Towards an Economic Model}

- Pooled mining and infrastructure investment.

- A history of pooled mining, and what we can infer empirically about the motivations of miners? Prefer low variance.

