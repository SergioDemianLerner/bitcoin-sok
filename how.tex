\section{An Introduction to Bitcoin}

Bitcoin's underlying mechanism.

Mining, transaction propagation.

Introduction of currency. Use of exchanges, market places. Financial ecosystem.


\subsection{System Attack Model and Consensus Requirements}

The Bitcoin mining consensus proof from the whitepaper, assuming a majority of the hash power is ``honest''. Bitcoin's solution to the consensus problem is novel, although the proof known so far is for the wrong model (honest rather than altruistic). Given a set of rules, in a distributed system, the problem remains of how to choose a correct order of operations.

Although, so far, the economic incentive system seems to work, there are only heuristics and no clear model. As best we can tell, the reasoning is circular. Incentive system is needed to make the system secure. The built-in currency that rewards participants has to be valuable for the incentive to be meaningful. The currency is valuable only if the system is secure. The ``death spiral.''


\subsection{Bitcoin as an Abstract Platform}

What functionality does Bitcoin provide in the abstract? Originally presented as a mechanism for online ``payments,'' in the abstract it is a much more versatile platform.

The following is an abstract description of Bitcoin's goals, independent of system assumptions or implementation:
\begin{itemize}
\item (Append-only Log with Stabilizing-Consistency) The network eventually stabilizes to a single global view of a linear ordering of transactions. Unlike standard consistency, there is not necessarily a final time at which it's guaranteed to be settled. The more time passes, the more likely that a current view is settled. Given a model of attacker strength, and security probability, one can calculate an amount of time to wait before considering a prefix to be authoritative.
\item (Validation) Only valid transactions are committed to the log, where 'validity' is a function of the prior history of transactions.
\item (Fairness) After expending sufficient resources (i.e., by paying a fee), any transaction is eventually committed to the log.
\end{itemize}

Stabilizing consistency is weaker than a typical distributed system. In a typical distributed system, you receive an acknowledgment at some finite time that the transaction has been committed. In Bitcoin, there is no such acknowledgment, and users must user their own discretion about how long to wait to consider a transaction committed.

A typical definition of fairness or liveness would specify that *any* transaction should eventually accepted, without reference to adequate payment. However, since Bitcoin operates in a model with no established identities, this would be impossible due to the ability for an attacker to create sybil identities and flood the system.

The view of Bitcoin as a platform is that the currency is only inherently useful to the extent that it allows you to pay for transaction fees. In this sense it is comparable to postage stamps.


\subsection{Bitcoin's Model Assumptions}

Bitcoin's assumptions are weaker than in a standard distributed system - most notably, there are no pre-established identities. On the other hand, the assumptions are about the rational preferences of population - specifically that they respond to incentives, even when the incentives are inconclusive. Seems to be circular?


\subsection{How Changes are Applied}

Discussion of governance and the need for social out-of-band consensus to agree on rule changes. Soft forks, hard forks, and policy. Very few hard forks in history.

Discussion on the forums and mailing list that provide.

Ability for altcoins to develop.


\subsection{Towards an Economic Model}

Pooled mining and infrastructure investment.

A history of pooled mining, and what we can infer empirically about the motivations of miners? Prefer low variance.
