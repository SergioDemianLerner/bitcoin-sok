\section{ Exploration of the Cryptocurrency design space }

Since Bitcoin's inception, a vibrant community of participants and open source developers, as well as a number of researchers, have suggested improvements and extensions. We have developed a framework to aggregate and systematically evaluate these contributions. 

Two main categorizations stand out: a) there are proposals for protocols that use Bitcoin (or a variant) as a platform, and b) there are proposals for modifications to the platform itself. We treat these separately.

\subsection{ Source Collection: Technical Discussions from the Bitcoin Community}
Technical discussions and formal presentation of Bitcoin extension ideas are primarily conducted through the following venues:
\begin{itemize}
\item Bitcointalk forums (\url{https://bitcointalk.org/})
\item IRC Channels (\url{irc://freenode.net/#bitcoin-dev})
\item Developer mailing list hosted on sourceforge
\item Bitcoin wiki
\item Bitcoin Improvement Proposals
\end{itemize}
Our methodology involves collecting ideas from these and analyzing them.

As a resource to researchers, the discussions and ideas discussed in these venues represent a) what is desired about Bitcoin from its participants, b) what concerns or as of yet unexplained, c) what is thought to be possible or difficult, tradeoffs, which provide starting points for other research, and d) what facilitates actual deployment.

\subsection{Categorization of Extensions}

\subsubsection{Design Tradeoffs}
The need to compete with other similar cryptocurrencies establishes the essential tension underlying design decisions and tradeoffs. Bitcoin must encourage as much mining participation as possible in order to defend against powerful attackers. The main mechanism for encouraging miner participation is disbursing rewards denominated in Bitcoins, the internal unit of currency. Thus it is essential for that the application service provided by Bitcoin is useful enough that users are willing to pay usage fees and purchase currency from the miner. If an alternate cryptocurrency offers more features and appeals to more users, then it may compete with Bitcoin for miner participation, thus weakening its security. Because of this, Bitcoin may be expected to incorporating new features and functionalities if they appear to have sufficient user demand.

On the other hand, new features and functionality typically involve increased cost (in either, communication, storage, or computation) to participants running ``fully-validating'' nodes, which is a requirement for correct mining. It is unclear how much additional burden can be shouldered by the network without a loss of security.

A straightforward approach to reducing the costs of additional functionality is to delegate control to a trusted third party. This is axiomatically discouraged in the design goals of Bitcoin, as it constitutes an ``existential threat'' that increases the risk of an overall system failure, despite immediate benefits to efficiency and convenience.

Suggested improvements to Bitcoin generally fall into the following categories:
\begin{itemize}
\item Efficiency improvements
\item Increased stability and security (against various threats to the overall network, or economic stability)
\item Additional functionality, to enable new uses.
\end{itemize}

We prefer to distinguish between two kinds of security: as it concerns the health/stability of network system overall (e.g., as concerns Proof-of-Stake, Block times) and security of users (e.g., ZeroCoin, better keys or password protection).

It's worth identifying that there are many roles in Bitcoin.
\begin{itemize}
\item Participation as a node, i.e. relaying transactions and blocks
\item Mining, including participation in a mining pool
\item Exchanging, mixing, mirroring, merchants, and other services interacting with the currency itself.
\end{itemize}
A desirable feature may impact some roles but not others. For example, transaction privacy (e.g., Zerocoin) increases the security users but has no direct impact on miners. Likewise, additional costs to miners (or validating nodes) do not necessarily impact users directly.

Due to the need for governance, everyone to agree on the constitutional rules, there are several routes to deploying improvements to Bitcoin. Most proposed extensions to Bitcoin are essentially independent of the mechanism by which they could be integrated. For example, although Zerocoin is proposed as an extension to Bitcoin, it could also be implemented as a concurrent and independent protocol (an ``altcoin''), or as a third party service that Bitcoin users interact with. Other arrangements are possible as well, such as implementing a proposal as an overlay on top of Bitcoin. Another approach, considered a ``gradual soft fork'' involves having the protocol be voluntarily included by miners. If the network of miners appear to reach a de facto consensus on a new extension, it would be easier politically to incorporate this in the mandatory rules. The following are ways in which an extension may be deployed:

\begin{itemize}
\item Hard-forking ``mandatory'' change, blocks without the modification will be rejected
\item Soft-forking changes, for subsets of existing possibility, miners can include cooperating data backward-compatibly with blocks.
\item Peer-to-peer network changes.
\item Optional client changes only, compatible with existing p2p network.
\item An external ``oracle,'' implemented as a trusted third party (or shared/quorum of third parties) recognizable by their public key signatures
\item An altcoin, a separate concurrent protocol
\end{itemize}

This imposes a hierarchy of extensions. In our terminology, an extension to Bitcoin may require only modifications to the client - anything that modifies how a user interacts with Bitcoin can count as a modification. We do mean to rule out things like ATMs that interact with the ordinary Bitcoin client. Bitcoin, even its current form, supports a number of extensible uses. Enables protocols between individuals, use Bitcoin in some way.

On one hand, the reference client accounts for the largest number of Bitcoin nodes. However as many users turn to hosted bitcoin wallets, it is hard to assign an actual user number to these.


\section{ Protocols using Bitcoin as a Platform }

The standard usage of Bitcoin is for financial transactions, in particular payments, denominated in the built-in currency. However there are a variety of more complicated financial transactions, including the construction of other currencies overlayed, as well as other uses, such as timestamping.

Protocols using Bitcoin as a platform can be evaluated in several ways:
\begin{itemize}
\item {\em Validation Costs.} The computational resource burden imposed on the network. This primarily consists of a) the total amount of communications that must be broadcast to the network (i.e., the size and number of transactions involved), b) the amount of storage that the transaction places in the replicated index, and c) the cost of validating a transaction. We analyze these according to the asymptotic cost, per transaction. In addition to the per-contract cost, there may also be an overall cost independent of the per-transaction cost; we indicate this cost in brackets.

\item {\em Party Costs.} Parties that participate in an outer protocol using Bitcoin as a component may have to provide their own additional communication resources as well. This includes local storage, computation, and communication (with either peers in the Bitcoin network or the other parties of the outer party), as well as the amount of time required before the transaction is complete. As with validation, there may be a baseline cost independent of the per-transaction cost; this is indicated in brackets.

\item {\em Trust Model.} While the Bitcoin network itself does not rely on trust in any designated party, outer protocols using Bitcoin are often characterized by leveraging or relying on trust in one of the parties. We indicate these by codes which will be described in the sections 

\item {\em Fork Required?} What modifications to the existing Bitcoin protocol (if any) are required.
\end{itemize}

A summary of this comparison, applied to the proposals we have collected, is provided in Table \ref{tbl:protocols}.

\subsection{Descriptions of the Rows}

\subsubsection{Purchases}
Purchases involve the transfer or exchange of Bitcoin as money, wherein a Buyer orders a good from a Merchant, pays, and then the Merchant delivers the good. {\bf Merc} indicates that the Merchant must be trusted to complete the delivery; {\bf 3rd} indicates that a third party, separate from the Buyer and Merchant, is trusted to intermediate.
\begin{itemize}
\item (Purchase, SPV) In a standard Bitcoin transaction, a Buyer transfers a quantity of BTC to a Seller by publishing a transaction referring to a coin currently owned by the Buyer, and rededicating it to a public key owned by the Seller. The Seller waits for some number of confirmations, until he is confident that the block containing the transaction will not be revised, and then performs an irrevocable action, such as shipping an item to the Buyer. To check that the transaction is valid before including it in a block, a miner checks one public key signature, and queries the index of unspent coins, which takes $O(log m)$ time overall, where $m$ is the size of the index. The Seller does not need to fetch the entire contents of each block - it is sufficient only to check just the proof-of-work, since it is assumed that the hashpower of the honest participants on the network is steady, and only mine valid blocks - this is an example of SPV security. Note that the Buyer must trust the Seller to fulfill his end of the arrangement.

\item (Escrow Purchase) If the Buyer does not wish to trust the Seller, but both parties can agree to trust a willing third party, then the third party can act as an Escrow agent. The Buyer constructs a transaction that allows the Escrow agent to choose whether to direct the funds to the Seller, or to the Buyer as a refund. The key idea is that the Escrow can only take these two options, and cannot take the funds for himself. Therefore the Escrow need not be fully trusted, but only must be trusted not to collude with either party. This model of Escrow is used by the BitMit auction site, as well as the former black market site SilkRoad. Independent Escrow provided by a company called ``BTCrow.'' The threat of damaged reputation may keep the Escrow honest. On the other hand, if there is a dispute, where both parties accuse the other of lying, the Escrow may essentially have no way to determine which is at fault.

\item (Time-locked Refund) The Escrow may abort and not sign any message, which would result in the funds being frozen indefinitely. Bitcoin supports a timelock mechanism, that allows a refund transaction to be prepared in case of the escrow agent timing out.
\item (Green Address~\cite{greenaddress}) If payment is received from a trusted party, then it may not be necessary to wait for confirmations. A Green Address refers to a publicly known address associated with a reputable entity, such as a hosted account provider (i.e., a bank). This is analogous to a cashier's check. If a double spend occurs, it would be easy to publish evidence of such, tarnishing the entity's reputation. MtGox has provided the option of withdrawing payments via a well-known Green address.

\item (Quick Purchase) Bitcoin vending machine ~\cite{bitcoinsnack} Also several companies are currently deploying machines exchanging Bitcoin for other currencies, in particular Lamassu and BitcoinATM. The challenge is in accepting the coins in a short period of time. The approach is to connect to a large number of nodes and attempt to detect double spends, in which case can enter a panic mode.

\item (Pay-to-Script-Hash) In the case of an escrow transaction, for example, the sender must create a transaction that includes the hashes of several public keys. The sender may have to pay a fee proportional to the size of the transaction. As a minor optimization, the sender can construct a transaction that includes just the hash of a larger script containing these public keys, and the full script must be provided by the receiver. This allows the sender to pay less in fees - the recipient may pay fewer fees overall to make the next transaction, if it is not time critical.~\cite{paytoscripthash}
\item (Micropayments) Suppose a Client wants to have a per-minute subscription for some service provided by a Server, and to decide whether to continue at each minute. Publishing a new transaction each minute would require a lot of transaction fees and storage on the network. Instead, the client creates a transaction that ``bails in'' the maximum amount of money for the day, along with a refund transaction that times out at the end of the day. After each minute, if the client wishes to continue, he transmits a new transaction to the server that increases the amount. When the server is ready to claim the transaction, he signs the largest transaction and publishes it. Only one transaction ever needs to be published, yet the Server is never able to take more than the Client offered.
\item (Balance query, auth index) So far, every transaction has involved only point-wise queries to the index. An example of a more complicated query would be to compute the total balance of an address, meaning the sum of all coins associated with a particular public key. This could be done using a linear scan of the unspent transaction outputs index, but this would be inefficient. Instead, a second index could be sorted according to public key (at least for standard transaction types). This would increase the storage cost to a validator by a constant factor. In any case, the user performing the query would be required to store the entire index as well. Alternately, suppose that the unspent transaction index is committed to as an authenticated data structure, the root of a merkle hash tree. Then the user can confirm the balance using only a $O(log m)$ size transmission from an {\em untrusted} node.
\end{itemize}

\subsubsection{Overlay Currencies}
Besides using Bitcoin as money, Bitcoin as a platform can also be used to maintain alternate currencies, such as personally-issued IOU coins, transferable shares in a Business, etc. A colored coin is introduced by fiat - an issuer, such as the executive of a corporation, declares that a particular quantity in the new protocol should be regarded as a currency. The protocols place restrictions on how these quantities are used in transactions, such as guaranteeing they are conserved.

\begin{itemize}
\item (MasterCoin~\cite{mastercoin}) MasterCoin proposes to create overlay currencies by building an entirely separate transaction protocol that uses Bitcoin as the underlying global append-only log. The observation is that arbitrary data packets can be embedded in Bitcoin transaction messages, and given a sequential ordering based on the contents of blocks. The first valid sequence of data packets is considered authoritative. The biggest drawback to this approach is cost; because Bitcoin is used only as an append-only log, miners do not perform any validation. The validity of a particular data packet cannot be determined except with knowledge of the previous history of transactions. Therefore determining whether a MasterCoin packet represents a valid coin requires processing every Bitcoin transaction in order to determine if it is a Mastercoin transaction, and maintaining a sufficient index.

\item (Colored~\cite{coloredcoins,bitcointalk-coloredcoins} and SmartCoins~\cite{jgarzik-smartcoin}) Colored coins have similar goals to MasterCoin. The difference is that colored coins make use of the transaction structure of ordinary Bitcoin, rather than using a separate data protocol. Each unit of a colored coin is ``carried'' by units of Bitcoins. There is thus some corresponding notion of SPV security. It can only be valid if every input corresponds to a previous valid transaction, etc. However, in order to determine the color of the coin, you must traverse the entire relevant subgraph of transactions, bounded by $n_{\cal C}$, the number of transactions associated with color ${\cal C}$.

\item (Freimarkets~\cite{freimarkets} and Ripple) Whereas Colored Coins and Master Coin involve the creation of overlay currencies using, Ripple proposes to build the rules governing such overlay currencies directly into the validation rules. This increases the cost of validation, however it restores the ability to use SPV security, where users do not need to store the entire index. Note that Ripple also proposes a different consensus model based on designated trusted entities rather than incentivized proof-of-work.

%\item (OpenTransactions) A simpler approach to overlay currencies is to maintain the overlay currencies using a separate third-party server. Arguably such a scheme has little to do with Bitcoin, except that the alternate currencies can be exchanged.
\end{itemize}

\subsubsection{Private Transactions}
Standard Bitcoin transactions are published in the clear, although account numbers are typically not associated with any real world identity. Publishing transactions makes it easy to verify that system invariants are maintained, for example that the total amount of currency is conserved. However, many users would prefer their financial transactions to be kept private. There are several approaches to obscuring relations between keys, essentially by performing a mix of some kind. The simplest solution involves a third party server that is trusted to delete its logs. On the other hand, the following two approaches enable secure coin-mixing without the involvement of a third party.

\begin{itemize}
%% \item (Mixers) The simplest (and most common) approach is for parties to transfer equal amounts of coins to a trusted third party mixer, who shuffles the coins, returns them to their owners, and deletes its log. Using Chaumian blinded signatures, users may not need to trust the mixer to delete logs, although they will still need to trust the mixer not to abscond with their funds. Using zero knowledge techniques, the operation of a third party mixer can be made publicly verifiable, which further reduces the likelihood of the mix stealing funds without getting caught.
\item (CoinJoin) A shuffle of $n$ transaction inputs and $n$ transaction outputs can be conducted atomically using a single transaction. However, this requires several rounds after which each party signs the entire transaction. Notably, it is easy for an attacker to prevent the mix from completing.
\item (ZeroCoin)~\cite{miers2013zerocoin} Perhaps the most thorough solution is to embed the functionality of a publicly verifiable (and public coin) third party mix directly into the validation rules of the block chain. This incurs additional validation cost.
%% \item (Aggregate-Signature CoinJoin) Using aggregate signatures, the transaction can be signed incrementally, reducing the amount of interaction needed to perform an atomic mix.
\end{itemize}

\subsubsection{Credentials}

\begin{itemize}
%% \item (Bitcoin-OTC) As mentioned before, despite the lack of trusted entities required for money transfers in Bitcoin, even a simple purchase requires some degree of trust between a Buyer and a Seller. Reputation aggregation systems, similar to the ones provided by eBay and Amazon, can be used to motivate honesty and discourage damaging a reputation. A popular reputation aggregation system for the Bitcoin community is run as a third party service. Conceptually a similar system could be modified to run within the Bitcoin system itself.
\item (Namecoin) An altcoin has been proposed to serve the functionality of the current Domain Name System. The validation rules are augmented to include updates of domain name records. Unused domain names cost a minimal fee to register initially, but the owner of a domain name can sell it (atomically) for an arbitrary price. Periodic renewal fees discourage squatting of domain names.
\item (Fidelity Bond) A general approach to meaningful identities in an otherwise anonymous environment is to require something of value invested in each identity. By burning or encumbering a quantity of money associated with a particular name, a party can signal that they value the reputation associated with this identity.
\item (Anonymous Credentials) The {\bf Issuer} code here indicates that the root identity issuer must be trusted to assign identities correctly.
\end{itemize}

\subsubsection{Cross-Chain Transactions}
Bitcoin is not the sole cryptocurrency, instead it must be understood as the currently-dominant member of a potentially unbounded ecosystem of possibly competing or cooperating crypto-currencies. The simplest interaction between two related currencies is akin to a foreign currency exchange.

\begin{itemize}
\item (TierNolan's Protocol~\cite{tiernolan}) There exists a three phase process by which two currencies can be atomically exchanged
\item (P2PTradeX~\cite{p2ptradex}) Another proposal involves having one currency validate blocks in the other. This potentially allows for more complicated interactions, as validators of one currency essentially enjoy a reduced-cost SPV view of the other chain, and vice versa. (example needed)
\end{itemize}

\subsubsection{Other}
\begin{itemize}
\item (Coin Flip Lotteries) Using Bitcoin has a clear appeal to online gambling sites (including illicit ones), since it allows users to cash in and out without having to interact with ordinary banking institutions. While several gambling sites feature games like poker\anote{cite here} or elaborate virtual casinos~\anote{cite minecraft bitcoin casino}, some of the most popular have been very simple roulette games, such as Satoshi-dice~\anote{cite satoshi dice}. \anote{TODO: add a comment about the trend of ``provable fairness'' in Bitcoin gambling, and the way that Satoshi-dice uses Bitcoin transactions}. Multi-player lotteries of this sort can also be conducted directly on top of the Bitcoin blockchain protocol, without any counterparty risk.~\cite{fairmultiparty}
%\item (Transaction Puzzle) Although in typical use, quantities of Bitcoin are associated with public keys such that matching signatures must be published in order to spend the coins, it is also possible to associate a quantity of Bitcoins with a puzzle (for example, finding the preimage of a hash), such that a solution to the puzzle must be published in order to claim the reward.
\item (CommitCoin) In CommitCoin, the hash of a document is embedded in the public key of a transaction that is published and timestamped in the blockchain. The user must store the document herself, and at any later time can publish the document and prove to anyone else that the hash of the document was known previously to the timestamped date.~\cite{commitcoin}
\end{itemize}

\newcommand{\riskpower}{K}
\newcommand{\overall}[1]{{~[ {#1} ]}}
\newcommand{\Oh}[1]{{#1}}

\begin{landscape}
%\begin{sidewaystable}
\begin{table}
\caption{Evaluation Summary of Protocols using Bitcoin as a Currency and/or a Transactional Platform ($m$ is the number of elements in the index, $n$ is the number of blocks for the duration of the protocol, $c$ is the number of transactions in a block}
\centering
\begin{tabular}{|r|r|r|r|r|r|r|r|r|r|r|r|}
\hline 
 & \multicolumn{3}{c|}{\textbf{Validation Cost Per Tx [Overall]}} & \multicolumn{4}{c|}{\textbf{Party Cost per Tx [Overall]}} &  && &\\
\hline
\textbf{Protocol} & \textbf{Comms} & \textbf{CPU} & \textbf{Space} & \textbf{Comms} & \textbf{CPU} & \textbf{Space} & \textbf{Time} & \textbf{Trust} & \textbf{Fork} & \textbf{Advantages} & \textbf{Disadvantages} \\

\hline \hline \textbf{Purchases} \\ 
\hline Standard &
$\Oh{1}$ & $\Oh{\log m}$ & $\Oh{1}$ & 
$\Oh{1} \overall{\Oh{nc}}$ & $\Oh{\log m} \overall{\Oh{nc \log m}}$ & $\Oh{1} \overall{\Oh{m}}$ & 1\riskpower & 
Merc & No & & \\

\hline SPV-security & 
$\Oh{1}$ & $\Oh{\log m}$ & $\Oh{1}$ & 
$\Oh{\log c} \overall{\Oh{n}}$ & $\Oh{\log c} \overall{\Oh{n}}$ & $\Oh{1} \overall{\Oh{1}}$ & 1\riskpower & 
Merc & No & Less party cost & \\

%% ``t*O(1)
%% "O(t+log m) work
%% "O(t) 
%% "O(t)
%% ``2($/P)
\hline 
Escrow (1 of t sigs) & 
$\Oh{t}$ & $\Oh{t+\log m}$ & $\Oh{t}$ &
$\Oh{\log c} \overall{\Oh{n}}$ & $\Oh{\log c} \overall{\Oh{n}}$ & $\Oh{1} \overall{\Oh{1}}$ & 2\riskpower & 
3rd & No & No Merc trust & Limited 3rd party\\

%% \hline 
%% P2SH Escrow & 
%% $\Oh{1}$ & $\Oh{t+\log m}$ & $\Oh{t}$ &
%% $\Oh{t}$ & `` & `` & 2\riskpower & 
%% `` & No &  $$ \\

\hline Green Address & 
$\Oh{1}$ & $\Oh{\log m}$ & $\Oh{1}$ & 
$\Oh{\log c} \overall{\Oh{n}}$ & $\Oh{\log c} \overall{\Oh{n}}$ & $\Oh{1} \overall{\Oh{1}}$ & $\Oh{1}$ & 
Merc,3rd & No & Instant confirm & 3rd party trust\\


\hline Micropayments & 
$\Oh{1/f}$ & $\Oh{1/f \log m}$ & $\Oh{1/f}$ & 
$\Oh{\log c} \overall{\Oh{n}}$ & $\Oh{\log c} \overall{\Oh{n}}$ & $\Oh{1} \overall{\Oh{1}}$ & 2\riskpower & 
 None & No & Low cost, trust & Incr. goods \\

%\hline Balance Query (Auth) & \\

\hline \hline \textbf{Overlay Coins} \\
\hline MasterCoin & 
$\Oh{1}$ & $\Oh{\log m}$ & $\Oh{1}$ & 
$\Oh{1} \overall{\Oh{nc}}$ & $\Oh{\log m} \overall{\Oh{nc_{\cal M} \log m_{\cal M}}}$ & $\Oh{1} \overall{\Oh{m_{\cal M}}}$ & 1\riskpower & 
 Issuer & No & & High overhead \\

\hline Colored/Smart Coins & 
$\Oh{1}$ & $\Oh{\log m}$ & $\Oh{1}$ & 
$\Oh{n_{\cal C} + \log c} \overall{\Oh{n}}$ & $\Oh{n_{\cal C} + \log c} \overall{\Oh{n}}$ & $\Oh{1} \overall{\Oh{1}}$ & 1\riskpower & 
 Issuer & Lower overhead & & \\

\hline Freimarkets & 
$\Oh{1}$ & $\Oh{\log m}$ & $\Oh{1}$ & 
$\Oh{\log c} \overall{\Oh{n}}$ & $\Oh{\log c} \overall{\Oh{n}}$ & $\Oh{1} \overall{\Oh{1}}$ & 1\riskpower & 
 Issuer & Yes & Low overhead & Fork required \\

%\hline OpenTransactions & \\

%\hline ChronoBit & \\
%\hline Tribute to Len & \\
%\hline OP\_DROP & \\

\hline \hline \textbf{Private Transactions} \\ 
\hline Zerocoin & 
$\Oh{1}$ & $\Oh{\log m + \log n}$ & $\Oh{1} \overall{m+n}$ & 
$\Oh{\log c} \overall{\Oh{n}}$ & $\Oh{\log c} \overall{\Oh{n}}$ & $\Oh{1} \overall{\Oh{1}}$ & 2\riskpower + w &
 None & Yes & & High cost \\

\hline CoinJoin & 
$\Oh{q}$ & $\Oh{q \log m}$ & $\Oh{q}$ & 
$\Oh{q + \log c} \overall{\Oh{n}}$ & $\Oh{q + \log c} \overall{\Oh{n}}$ & $\Oh{q} \overall{\Oh{1}}$ & 1\riskpower & 
 None & No & & DoS \\

%% \hline Agg. Sig. CoinJoin & 
%% $\Oh{1}$ & $\Oh{\log m}$ & $\Oh{1}$ & 
%% $\Oh{\log c} \overall{\Oh{n}}$ & $\Oh{\log c} \overall{\Oh{n}}$ & $\Oh{1} \overall{\Oh{1}}$ & 1\riskpower & 
%%  & & \\

%% \hline External Mixers & 
%% $\Oh{1}$ & $\Oh{\log m}$ & $\Oh{1}$ & 
%% $\Oh{\log c} \overall{\Oh{n}}$ & $\Oh{\log c} \overall{\Oh{n}}$ & $\Oh{1} \overall{\Oh{1}}$ & 1\riskpower & 
%%  & & \\

\hline \hline \textbf{Credentials} \\
\hline Namecoin &
$\Oh{1}$ & $\Oh{\log m}$ & $\Oh{1}$ & 
$\Oh{\log c} \overall{\Oh{n}}$ & $\Oh{\log c} \overall{\Oh{n}}$ & $\Oh{1} \overall{\Oh{1}}$ & 1\riskpower & 
 None & Yes & & \\

\hline Fidelity Bond & 
$\Oh{1}$ & $\Oh{\log m}$ & $\Oh{1}$ & 
$\Oh{n + \log c} \overall{\Oh{n}}$ & $\Oh{n + \log c} \overall{\Oh{n}}$ & $\Oh{1} \overall{\Oh{1}}$ & 1\riskpower &
 None & No & & Expensive \\

\hline Anonymous Credentials & 
$\Oh{1}$ & $\Oh{\log m}$ & $\Oh{1}$ & 
$\Oh{\log c} \overall{\Oh{n}}$ & $\Oh{nc + \log c} \overall{\Oh{n}}$ & $\Oh{1} \overall{\Oh{1}}$ & 1\riskpower & 
 PKI & No & & Scanning cost \\

\hline \hline \textbf{Other} \\ 
\hline CommitCoin & 
$\Oh{1}$ & $\Oh{\log m}$ & $\Oh{1}$ & 
$\Oh{d + n + \log c} \overall{\Oh{n}}$ & $\Oh{n + \log c} \overall{\Oh{n}}$ & $\Oh{d} \overall{\Oh{1}}$ & 1\riskpower & 
 None & No & & \\

\hline Coin Flip & 
$\Oh{1}$ & $\Oh{\log m}$ & $\Oh{1}$ & 
$\Oh{\log c} \overall{\Oh{n}}$ & $\Oh{\log c} \overall{\Oh{n}}$ & $\Oh{1} \overall{\Oh{1}}$ & 2\riskpower & 
 Pub. Verif. & No & & \\

\hline Transaction Puzzle & 
$\Oh{1}$ & $\Oh{\log m}$ & $\Oh{1}$ & 
$\Oh{1} \overall{\Oh{1}}$ & $\Oh{w} \overall{1}$ & $\Oh{1} \overall{\Oh{1}}$ & $\Oh{w + 1/\alpha}$ & 
 None & No & & Miner only \\

%\hline Vanity Address & \\

\hline \hline \textbf{Cross-Chain Swap} \\ 
\hline TierNolan's protocol & 
$\Oh{1}$ & $\Oh{\log m}$ & $\Oh{1}$ & 
$\Oh{\log c} \overall{\Oh{n}}$ & $\Oh{\log c} \overall{\Oh{n}}$ & $\Oh{1} \overall{\Oh{1}}$ & 3\riskpower & 
None & No & & \\

\hline P2PTradeX & 
$\Oh{1}$ & $\Oh{\log m}$ & $\Oh{1}$ & 
$\Oh{\log c} \overall{\Oh{n}}$ & $\Oh{\log c} \overall{\Oh{n}}$ & $\Oh{1} \overall{\Oh{1}}$ & 2\riskpower &
None & Alt & & Altcoin only \\
\hline

\end{tabular}
\end{table}
%\end{sidewaystable}
\label{tbl:protocols}
\end{landscape}


\section{Proposed Modifications to the Platform}

Many proposed modifications to Bitcoin are meaningful independent of any particular protocol. Generally these improve security, decrease the cost of participation, or increase the speed at which transactions can be included. Often there is a tradeoff, for example increased security at the cost of. Many involve changing parameters which are configured in Bitcoin without any particular justification.


\subsection{Description of the Columns}

Proposed modifications to the Bitcoin platform can be evaluated according to the following dimensions.

\begin{itemize}
\item (System efficiency) Does it decrease the cost (storage, computation, communication) of transaction or block validation?
\item (Economic structure) Besides efficiency, does it alter the reward for miners to participate?
\item (User Centric) For users of the system (participating in protocols involving transactions or queries to the system) does it increase their security? Or reduce their cost of participation?
\item (Security Model) Does it reflect a change or refinement of the participation or overall attack model?
\item (Integration Requirement) Does it require a change to the block validation rules (hard fork), or the transaction inclusion rules (soft fork), the peer-to-peer messaging protocol, or just the client?
  \begin{itemize}
    \item (Hard-Fork) For new functionality that existing full-validating clients (whether or not they also mine) will reject. If miners switch to this, cause disagreement between full nodes and SPV nodes. Hard-fork requires every validating node to upgrade their clients immediately.
    \item (Soft-Fork) Refinements of existing behavior, for example disabling an opcode, enforcing a subset of the transaction script language. A majority of mining participants can enforce a soft-fork by refusing to work on blocks violating the new rule.
    \item (Miner Policy) Miners can choose not to include transactions into their own blocks, yet still accept blocks made by other miners. IsStandard and eligius. Transaction prioritization.
    \item (P2P) Affects the messaging layer. How nodes find each other, announce, request, transfer data such as blocks and transactions. How to serve old blocks to new nodes on the network.
    \item (Client) Only requires a change of the client.
  \end{itemize}
\item (Maturity): is there an implementation available? Has it been implemented in an altcoin? Or has it been adopted as a patch to Bitcoin? Any particular change can be made as an altcoin.
\end{itemize}

\subsection{Description of the Rows}

\subsubsection{Hard-Fork Parameters}
The outermost layer of the system.
\begin{itemize}
\item (Difficulty adjustment) Time is a critical parameter. If this is set too low, then the network does not reach consensus, or the effective power of the network against an attacker is diminished. If it is set too high, then the variance for measuring the amount of security is lower. This is set by magic parameter. How else should it be set?
\item (Alternate Mining Systems) Alternate functions based on hash function, such as scrypt. Useful proof-of-work. Proof-of-stake.
  \begin{itemize}
  \item (Proof-of-Stake) Alternate security model. Perceived possibility of an economic attack.
  \item (Designated authorities) Trusted (or semi-trusted) servers may be designated to receive, publish, and linearize transactions ~\cite{lauriecoin}. Another point out is Ripple's approach to unique node lists. We do not understand the model under which these are secure. On the other hand, we don't yet have a clear model in which Bitcoin's incentives work either.
  \item (Useful proof of work) Bitcoin appears to be wasteful. There are two ways in which it is wasteful - one is that the energy burned is not directly useful. A common suggestion is to replace the proof of work with a search function, such that a solution would be useful, distributed search project. It has been argued (e.g., by Kroll et al) that this must be a pure public good, or else it subsidizes an attacker. Another form of waste is that the investment in more efficient mining devices has led to the development and distribution of devices that are useful for computing SHA2 and nothing else. Permacoin: in concurrent work we investigate the potential to replace the puzzle with a ``storage hard'' puzzle, so that mining equipment is copies of a dataset.~\cite{permacoin}
    \end{itemize}
\item (Block Size Limit) Remove it all together? \anote{Should summarize Petertodd's arguments here}
\item (Monetary Policy) Bonus schedule. Demurrage. Coinbase maturity. Conservation invariant.
\item (Transaction Scripts) New op codes, including ones disabled. Other signature schemes. Access to the index? Suspended computation. Ability to refer to other scripts. Turing complete? Succinct proofs with TinyRAM/Pinocchio? Access to additional indexes?
\end{itemize}

Not necessarily have to be linear to be useful, although some conflicts such as a double-spend require linearization.

\subsubsection{Mining Behavior}
Although miners are bound in many ways by the agreed-upon rules --- no miner can commit an invalid block, for example, since other miners will reject it --- miners are also given a lot of discretion, especially in the form of choosing which transactions to include in a block, and which transactions to relay. In particular, a miner can choose to ignore a transaction, and it's difficult for other miners to punish them --- perhaps the miner legitimately never received the transaction in the first place. On the other hand, miners tend to follow common conventions (known as ``miner policies'' or ``soft-fork rules'') and can in general be expected to respond to rational incentives. The simplest example of miner policy is prioritizing transactions that carry fees. Proposed variations of miner policies are enumerated as follows:
\begin{itemize}
\item (Transaction Preference) Priority. Fees. Dust discouragement. Refusing to mine on non-standard transactions (policy change). For example, the Eligius mining pool mines nonstandard transactions. Probabilistic transaction validation.
\item (Block Preference) No standard client has proposed anything other than take the longest block, and most recent one found if there is a tie. Kroll et al. argue that taking the longest block is a focal point. Miners may decide to refuse mining on blocks containing transactions they don't like. If a miner deviates from the consensus here, then it will waste its mining power. If enough take it, then it is a ``soft-fork rule change,'' can be used to enforce a subset of the rules. 
\item (Computationally Secure Verification) Commitment of index. Cryptographic improvements for validation. Efficiency Pinocchio or TinyRAM. Zero knowledge validation of commitments? ZeroCoin is a limited form of this. Compressed work sample.
\item (Pool participation) Forms of rewards. P2Pool, pool hopping (cite meni rosenfeld), reward schemes. Apparent desire for low variance. Stratum mining protocol and getwork.
\end{itemize}

\subsubsection{P2P System}
The system that passes around blocks and transactions. Primarily concerned with denial of service, while maintaining connectivity. Also concerned with privacy. Conceptually the entire P2P network could change, or a new P2P network could be developed, that interfaces with other Bitcoin nodes.
\begin{itemize}
\item (Relay Transaction/Block Preferences) Similar to mining behavior, although applies to other nodes not just miners.
\item (Header only block download) \anote{TODO: summarize}~\cite{headerssync}.
\item (Faster block relaying) Fast block propagation is crucial for the network to converge on recent blocks. Christian Decker et al. have studied block propagation and proposed faster variants~\cite{decker2013information}; there is also an earlier implementation by Luke-Jr\footnote{Source code commit by Luke-Jr,\url{https://gitorious.org/bitcoin/luke-jr-bitcoin/commit/4e54ea804ccdd2223e622497f0d46cceb27b9d22}}. Greg Maxwell proposed network coding as a method to reduce the cost of transmitting blocks, assuming that a peer already knows some portion of the block.~\cite{blocknetworkcoding}
\item (Client puzzles) Denial of Service is a common problem in networks, although fairly few of the known countermeasures are currently implemented in Bitcoin. Bitcoin nodes are essentially open to strangers, and are therefore vulnerable to resource exhaustion attacks. One countermeasure is to have a potential client solve a small proof-of-work before allocating resources. Ordinarily, such puzzles represent unrecoverable waste; on the other hand, since the Bitcoin network already makes use of puzzles, a natural approach is to ask potential clients to perform a small amount of Bitcoin mining work before allocating resources, thereby getting the benefit with no additional waste --- this general approach is known as a bread-pudding protocol.~\cite{breadpudding}
\end{itemize}

\subsubsection{Client/Agent}
\begin{itemize}
%\item (Blockchain/Index management) Pruning is described in the Bitcoin whitepaper. and Ultraprune~\cite{}. LevelDB and database change.
\item (SPV vs Full Node Security) 
\item (Wallet, key storage) Encrypted wallet, HD wallet, brain wallet, paper wallet, etc.
\item (Transaction Construction) Making change. This is where transaction anonymity is attacked, linking keys. Change transaction.
\item (Hardware clients) Although most end users interface with Bitcoin using clients for general purpose computers or mobile devices, storing private keys on specialized hardware such as smart cards or secure tokens have several potential advantages. These devices may be made tamper-proof, can be kept in the user's (physical) wallet or keychain, aren't vulnerable to computer viruses affecting general purpose operating systems, need not connect directly to the internet, and may even present a more convenient user interface. Bitcoin hardware wallets include the TREZOR\footnote{TREZOR product website, \url{http://www.bitcointrezor.com/}}. Other specialized hardware Bitcoin clients include vending machines~\cite{bitcoinsnack} or ATMs~\footnote{See for example Robocoin, Genesis1, and Lamassu\anote{cites needed here}}.
%\item (Confirmations and Decision Support) In a standard ``purchase'' use case, a merchant receives payment Waiting 6 blocks for confirmation, or estimating an attack. Showing confirmations to the user. Snack and vending machine. SPV Security.
\end{itemize}


