\section{Discussion}

\subsection{Smart Contracts a Universal Platform}
- Older documents from the cypherpunk era about smart contracts as a general abstraction

-- Mark Miller's thesis, and ``Ode to the Granovetter'' paper about capability-based financial contracts \cite{miller2001capability}

-- Nick Szabo's papers on smart contracts, coining the word. \cite{szabo1997formalizing}

- Current attempts towards smart contracts, including Ethereum, OP\_EVAL, and Mike Hearn's wiki page about bonds and financial contracts~\cite{bitcointalk-bondmarkets}.

\subsection{Interactions between Bitcoin and competitors}
One of the murkiest areas is understanding Bitcoin's security in the context of the larger ecosystem of similar cryptocurrencies, where multiple separate and concurrent protocols compete for the same computing resources. Bitcoin's security model essentially requires an amount of mining participation larger than any attacker. Bitcoin's incentive mechanism functions similarly to a fundraiser or a recruitment drive; the more participation, the more an attacker would have to pay to defeat it. This approach to security could be described as safety in numbers. However an implication is that for Bitcoin to be secure, it must stay on top, and attract the greatest amount of participation. \footnote{For example, in 2012 a Bitcoin mining pool attacked a small alt-coin, CoiledCoin. A minor participant on the larger Bitcoin network could easily join a smaller network and overwhelm it. }
%\url{https://bitcointalk.org/index.php?topic=56675.msg678006#msg678006}


\subsection{Bitcoin as money}
Three definitions are normally given as money. 1) Unit of account, 2) means of exchange, 3) store of value. Do these form design criteria for a system like Bitcoin? In other words, do these requirements form a problem statement to which a system like Bitcoin can be derived as a solution? Empirically, Bitcoin satisfies these. It has maintained a value over three years, it can be used bin exchanges, and accounts are denominated in BTC. However it isn't clear how we can argue that Bitcoin will maintain this. Are the other coins money as well?

The idea that money comes from barter can be interpreted as a matter of theory, rather than a historical account. Money is *reducible* to barter, in the sense that given the ability to exchange items of value (i.e., commodity goods), one of the goods naturally becomes used as ``money''. This is the approach taken in the Kiyotaki-Wright model of money. However, the other direction does not seem to hold - to the extent Bitcoin implements money, money is not sufficient to enable a barter. Trading virtual currency for mail order goods, seems to require  external mechanism of some kind, such as a trusted mediator, legal enforcement, or the threat of a tarnished reputation.

Through the lens of Bitcoin and the associated ecosystem of cryptocurrencies, we may have an unexpected opportunity to observe the emergent formation of money (or something similar) under different environment conditions.


\subsection{Remaining Open Questions}

How to theoretically model Bitcoin? Especially in a way that accounts for an ecosystem with multiple currencies cooperating or competing for participation.

How to match fees to costs?

How to have scalable participation, where the state is not replicated to the network in its entirety, but instead load-balanced in some way?
